\documentclass[12pt,a4paper,norsk]{article}
\usepackage[utf8]{inputenc}
\usepackage[norsk]{babel}
\usepackage{amssymb}
\usepackage{amsmath}
\usepackage{multicol}
\usepackage{appendix}
\usepackage{siunitx}
\usepackage[skip=5mm]{parskip}
\usepackage{randtext}
\usepackage{hyperref} % for hyperlenker mellom ref og label

\addto\captionsnorsk{%
  \renewcommand\appendixname{Appendiks}
  \renewcommand\appendixpagename{Appendikser}
  \renewcommand\appendixtocname{Appendikser}
}

\newcommand{\R}{\mathbb{R}}
\newcommand{\C}{\mathbb{C}}
\renewcommand{\P}{\mathbb{P}}
\newcommand{\B}{\mathbb{B}}
\newcommand{\imat}[1]{\left[\begin{smallmatrix}#1\end{smallmatrix}\right]} %inline col matrix
\newcommand{\overskrift}[1]{{\noindent\textbf{#1}}\nopagebreak\\[1em]\nopagebreak}
\newcommand{\mat}[2]{\left[\begin{array}{@{}#1@{}}#2\end{array}\right]}
\newcommand{\nullvec}{0}
\newcommand{\nullmat}{0}
\newcommand{\inner}[1]{\langle#1\rangle}

\DeclareMathOperator{\Sp}{Sp} %spenn

\DeclareMathOperator{\Col}{Col} %matriser
\DeclareMathOperator{\Row}{Row}
\DeclareMathOperator{\Null}{Null}
% \det finnes allerede

\DeclareMathOperator{\image}{im} %lineærtransformasjoner
\DeclareMathOperator{\rank}{rank}
% \ker finnes allerede

\DeclareMathOperator{\Ima}{Im} %Komplekse tall
\DeclareMathOperator{\Rea}{Re}

%vektorer
\newcommand{\vv}{\textbf{v}}
\newcommand{\vw}{\textbf{w}}
\newcommand{\vu}{\textbf{u}}
\newcommand{\vx}{\textbf{x}}
\newcommand{\vy}{\textbf{y}}
\newcommand{\vq}{\textbf{q}}
\newcommand{\vi}{\textbf{i}}
\newcommand{\vj}{\textbf{j}}
\newcommand{\vk}{\textbf{k}}

\title{TMA4115 --- Matte 3}
\newcommand{\lst}{krogstie}
\newcommand{\fst}{havard}
\author{Håvard Krogstie - \randomize{\lst{}.\fst{}@gmail.com} \\ Eivind Xu Djurhuus}
\date{\today}

\begin{document}

\maketitle

\noindent
Dette dokumentet er et forsøk på å samle det meste av definisjoner og resultater
som er kjekke å ha skrevet ned i Matte 3. Metoder på pensum blir gjennomgått,
med bevis der det later til å være pensum, altså forelesningsnotatene. Jeg har
prøvd å være mer rigorøs enn kildematerialet når det gjelder komplekse tall.
Det mest spennende i denne teksten står kanskje i resultatdelene, siden
det er der teoremer ofte dukker opp.

Dokumentet har jevnt over ikke kildehenvisninger, og er ikke vurdert av noen
med kunnskap i faget. Du finner heller ikke øvingskok her. Les på eget ansvar.
Kildekode og nyeste utgave på
\href{https://github.com/haved/notater/}{github.com/haved/notater}. Skrivefeil,
faktafeil, mangler eller uklaheter? Opprett en issue eller skriv en e-post!

\clearpage
\tableofcontents
\clearpage

\section{Lineær uavhengighet}
En mengde vektorer $v_{1}, v_{2}, \ldots, v_{n}$ er lineært uavhengige hvis og bare
hvis
\[c_{1}v_{1} + c_{2}v_{2} + \cdots + c_{n}v_{n} = 0 \hspace{4em} \Longleftrightarrow \hspace{4em} c_{1}=c_{2}=\cdots=c_{n}=0\]
Dette forutsetter at $+$ mellom vektorer og $\cdot$ med skalarer er definert. Vi
kaller venstresiden av ligningen for en lineærkombinasjon.

\subsection{Resultater}
For enhver lineærkombinasjon $u$ av lineært uavhengige vektorer
$v_{1}, v_{2}, \ldots, v_{n}$ finnes det kun ett sett med koefisienter
$c_{1}, c_{2}, \ldots, c_{n}$ som gir $u$.

\section{Vektorrom}
Et vektorrom består av en mengde vektorer $V$, en kropp med skalarer, en
addisjonsopperasjon $+$ mellom vektorer, og en multiplikasjonsopperasjon $\cdot$
mellom vektor og skalar. $+$ og $\cdot$ må være definert for alle $v \in V$, og skal
være lukket over $V$. 
Operajsonene følger følgende regler:
\begin{multicols}{2}
\begin{itemize}
  \setlength{\itemsep}{0em}
  \item $+$ er assosiativ
  \item $+$ er kommutativ
  \item $+$ har identitetselement: $\vec{0}$
  \item $+$ har invers
  \item $(ab) \cdot u = a \cdot (b \cdot u)$
  \item $\cdot$ har identitetselement: $1$
  \item $\cdot$ distribuerer over skalar-$+$
  \item $\cdot$ distribuerer over vektor-$+$
\end{itemize}
\end{multicols}

\subsection{Resultater}
Vi kan se at $0 \cdot u = 0$ og $(-1)u = -u$.
Det finnes kun én $1$, én $0$ og én $0$-vektor.

\subsection{Spenn}
Gitt en mengde vektorer $v_{1}, v_{2}, \ldots, v_{n}$ definerer vi spennet som
mengden av alle lineærkombinasjoner av vektorene. Vi bruker notasjonen
\[\Sp\{v_{1}, v_{2}, \ldots, v_{n}\} = \{c_{1}v_{1} + c_{2}v_{2} + \cdots + c_{n}v_{n}\ |\ c_{i} \text{
    er skalar}\ |\ i \in [1,n]\}\]
Mengden er lukket under vektoradisjon og
skalarmultiplikasjon, så det er et vektorrom (så lenge $+$ og $\cdot$ er definert
forsvarlig). Vi sier at vektorene $v_{1},v_{2}$ etc.\ spenner ut vektorrommet
$V$.

Merk at vektorene ikke nødvendigvis er lineært uavhengige, og at flere
mengder vektorer kan spenne ut det samme rommet.

\subsubsection{Resultater}
La $v_{1}, v_{2}, \ldots, v_{n} \in V$, da vil $\Sp\{v_{1}, v_{2}, \ldots, v_{n}\}$ være et
underrom av $V$.

\subsection{Eksempler}
$\R^{3}$, $\C^{2}$, $P_{2}$

\section{Basis}
En Basis $\B$ er en liste med \textbf{lineært uavhengige} vektorer som spenner ut et
vektorrom $V$. Alle basiser for rommet $V$ har samme antall vektorer, så vi
definerer dimensjonen på vektorrommet $\dim{V} = |\B|$.
Basiser lar oss bruke koordinatsystemer for alle vektorrom.

\subsection{Basiser og koordinatsystem}
For enhver basis $\B=(\vv_1, \vv_2, \cdots, \vv_n)$ til et n-dimensjonalt vektorrom $V$
og en vektor $\vv \in V$ vet vi at $\vv$ kan skrives som lineærkombinasjonen
\[\vv=c_1\vv_1 + c_2\vv_2 + \cdots + c_n\vv_n\]
Vi definerer \textit{koordinatvektoren} til $\vv$ i $\B$ som
\[[\vv]_\B=\mat{c}{c_1\\c_2\\ \vdots \\ c_n}\]
I tillegg kaller vi avbildningen 
\[\vv \mapsto [\vv]_\B\]
for \textit{koordinatsavbildningen} av $\vv$ med hensyn på $\B$.
Koordinatavbildingen er en funksjon $V \rightarrow \R^n$.
Avbildningen er en bijektiv lineærtransformasjon.

Den motsatte funksjonen er lineærkombinasjonen av basisvektorene
med koordinatvektoren som koeffisienter.
\[\mat{cccc}{\vv_1 & \vv_2 & \cdots & \vv_n}[\vv]_{\B} = \vv\]
For å finne koordinatvektoren løser man ligningen over for $[\vv]_{\B}$.

\subsection{Eksempler}
Det finnes uendelig mange basiser for alle vektorrom.
En basis for vektorrommet av andregradspolynomer: ($1$, $x$, $x^2$).
\subsubsection{Standardbasis}
For rommet $\R^{n}$ definerer vi standardbasisen til å være de $n$ vektorene
\[\mat{c}{1\\0\\0\\ \vdots \\ 0}, \mat{c}{0\\1\\0\\ \vdots \\ 0}, \mat{c}{0\\0\\1\\ \vdots \\ 0}, \cdots, \mat{c}{0\\0\\0\\ \vdots \\ 1}\]
Du kjenner dem kanskje igjen fra identitetsmatrisen.
Standardbasisen gir oss et koordinatsystem for $\R^n$ som er helt likt $\R^n$ selv.
Fysikkbøker skriver ofte vektorer i $\R^3$ som
\[5\vi + 8\vj -3\vk\]
Da brukes \vi, \vj og \vk som enhetsvektorene i $\R^3$, altså standardbasisen.
\[5\mat{c}{1 \\ 0 \\ 0} + 8\mat{c}{0 \\ 1 \\ 0} -3\mat{c}{0 \\ 0 \\ 1} = \mat{c}{5 \\ 8 \\ -3}\]

\subsection{Resultater}
\begin{itemize}
  \item Det er ikke mulig å ha $5$ lineært uavhengige vektorer i $\R^{4}$.
  \item Det er alltid mulig å lage en basis (Ta et spenn og fjærn lineært
    avhengige vektorer til det ikke er noen).
  \item $[\vv]_\B \in \R^{|\B|}$
  \item koordinatavbilding er en bijektiv lineærtransformasjon $T:V\Rightarrow \R^{\dim V}$.
\end{itemize}

\subsection{Skifte av basis}
Si at vi for et vektorrom $V$ har to basiser $\B$ og $\B'$. Av ulike grunner kan
det være lettere å arbeide med et av de to koordinatsystemene gitt av basisene.
Det vil derfor være ønskelig å finne ut hvordan koordinatvektorene $[\vv]_\B$ og
$[\vv]_{\B'}$ for en vilkårlig vektor $\vv \in V$ er relatert. 
Med andre ord, hvis du har en vektor $\vv$ utrykt med hensyn på en gammel basis $\B$,
hvordan vil man utrykke den samme vektoren med hensyn på en ny basis $\B'$? 
For å løse dette problemet finner vi først hvordan vi kan utrykke hver gamle
basisvektor som en lineærkombinasjon av den nye basisen. Dette tilsvarer å
finne, for hver vektor $\vu_i$ i $\B$, koordinatvektoren $[\vu_i]_{\B'}$
relativt til $\B'$. Vi kan så danne \textit{transisjonsmatrisen} (også kalt for
skifte-av-basis-matrisen)  $P_{\B\rightarrow\B'}$ fra $\B$ til $\B'$ slik: 
\[P_{\B\rightarrow\B'} = \mat{cccc}{[\vu_1]_{\B'} & [\vu_2]_{\B'} & \cdots & [\vu_n]_{\B'}}\]
Altså vi lager en matrise hvor hver kolonne er koordinatvektoren til en gammel
basisvektor utrykt med hensyn på den nye basisen. Denne matrisen vil ha egenskapen 
\[P_{\B\rightarrow\B'}[\vv]_\B = [\vv]_{\B'}\]

Et viktig teorem for transisjonsmatriser, som vi ikke skal bevise, er at
transisjonsmatrisen fra $\B'$ til $\B$ vil være inversen til transisjonsmatrisen
fra $\B$ til $\B'$ (og dermed at alle transisjonsmatriser er inverterbare). 
\[P_{\B'\rightarrow\B} = P_{\B\rightarrow\B'}^{-1}\]

I likhet med at vi kan utrykke vektorer med hensyn på flere basiser, kan vi også utrykke
lineære transformasjoner med hensyn på forskjellige basiser. Si at den lineære
transformasjonen $T$ kan representeres som matrisen $[T]_\B$ med hensyn på
basisen $\B$. Vi ønsker så utrykke $T$ som matrisen $[T]_{\B'}$ med hensyn på
$\B'$. Intuitivt vil metoden for dette være å skifte basis fra $\B'$ til $\B$, utføre den lineære transformasjonen som vi kan utrykke som matrisen $[T]_\B$,
og så skifte basis tilbake fra $\B$ til $\B'$. Fra dette får vi formelen 
\[[T]_{\B'}=P^{-1}[T]_\B P\]
hvor $P=P_{\B'\rightarrow\B}$.

\section{Underrom}
Et underrom $U \subseteq V$ er et vektorrom med vektorer fra $V$.
Vektoroperasjonene er de samme, og $U$ må være lukket over dem.

\subsection{Teste om underrom}
Sjekk at $U$ er lukket, altså at $u \in U \Longrightarrow a \cdot u \in U$ og
$u, v \in U \Longrightarrow (u+v) \in U$. \\
Underrommet må ha $0$-vektoren, $+$-inverser etc.

\subsection{Resultater}
$v_{1}, v_{2}, \ldots, v_{n} \in V \Longrightarrow \Sp\{v_{1}, v_{2}, \ldots, v_{n}\} \subseteq V$\\
Hvis $U \subseteq V$ og $\dim U = \dim V$, er $U = V$ \\
Gitt en lin.trans $T:V\rightarrow W$ vil $im T \subseteq W$ og $\ker T \subseteq V$

\section{Lineære transformasjoner}
En lineær transformasjon $T:V \rightarrow W$ (der $V$ og $W$ er vektorrom) er en funksjon
med to krav:
\[a \cdot T(v) = T(a\cdot v) \hspace{4em} T(v) + T(u) = T(v+u)\]
Vi kaller $V$ for domenet til $T$, og $W$ kalles kodomenet. De vektorene i $W$
det er mulig for lineærtransformasjonen å treffe kalles bildet til $T$, og
benevnes $\image T \subseteq W$. Vektorene i $V$ som blir til nullvektoren kalles
kjernen, og benevnes $\ker T \subseteq V$.
\begin{align*}
  \image T &= \{T(\vv) \in W\ |\ \vv \in V\} \\
  \ker T &= \{\vv \in V\ |\ T(\vv) = 0\}
\end{align*}

\subsection{Injektiv og surjektiv}
En lineærtransformasjon $T:V \rightarrow W$ er injektiv hvis
\[T(\vv) = T(\vu) \Rightarrow \vv = \vu\]
Dette er det samme som å si at enhver vektor $\vv \in V$ gir en ulik vektor
$T(\vv)$.
\[T \text{ er injektiv} \iff \dim \image T = \dim V \iff \ker T = \{0\}\]
Kun injektive funksjoner er entydig inverterbare.

En lineærtransformasjon $T:V \rightarrow W$ er surjektiv hvis
\[\forall \vu \in W\ \exists\vv \in V\ (T(\vv) = \vu)\]
Dette er det samme som å si at alle vektorer i kodomenet ($W$) er mulige å ``treffe''
med $T$.
\[T \text{ er surjektiv} \iff \image T = W\]

En funksjon som både er injektiv og surjektiv kalles bijektiv, eller
1-til-1-korrespondanse mellom domene og kodomene.

\subsection{Definisjon ut ifra basis}
Man kan definere en lineærtransformasjon $T:V\rightarrow W$ ut ifra hva som
skjer med vektorene i en basis for $V$. Gitt basisen $\B = (v_{1}, v_{2}, \ldots, v_{n})$
og avbildingene $T(v_{1}), T(v_{2}), \ldots, T(v_{n})$ kan vi transformere enhver
vektor $u \in V$. Finn først koeffisientene $c_{1}, c_{2}, \ldots, c_{n}$ til $u$ i $\B$.
\begin{align*}
  T(c_{1}v_{1} + c_{2}v_{2} + \cdots + c_{n}v_{n}) &= c_{1}T(v_{1}) + c_{2}T(v_{2})+ \cdots + c_{n}T(v_{n}) \\
  T(\mat{cccc}{v_{1} & v_{2} & \cdots & v_{n}}\mat{c}{c_{1} \\ c_{2} \\ \vdots \\ c_{n}}) &= \mat{cccc}{T(v_{1}) & T(v_{2}) & \cdots & T(v_{n})}\mat{c}{c_{1} \\ c_{2} \\ \vdots \\ c_{n}}
\end{align*}
Dersom basisen er standardbasisen for $\R_{n}$, altså at
$v_{1} = \imat{1 \\ 0 \\ \vdots}, v_{2} = \imat{0 \\ 1 \\ \vdots}$ etc.\ får vi at
\begin{align*}
  T(\mat{c}{c_{1} \\ c_{2} \\ \vdots \\ c_{n}}) &= \mat{cccc}{T(\mat{c}{1 \\ 0 \\ \vdots \\ 0}) & T(\mat{c}{0 \\ 1 \\ \vdots \\ 0}) & \cdots & T(\mat{c}{0 \\ 0 \\ \vdots \\ 1})}\mat{c}{c_{1} \\ c_{2} \\ \vdots \\ c_{n}}
\end{align*}
Vi har her gjort T om til en matrise. Alle lineærtransformasjoner kan gjøres om
til matriser, men vi må passe på hvilken basis matrisen transformerer fra og
til.

\subsection{Resultater}
For alle lineærtransformasjoner $T:V \rightarrow W$
\begin{itemize}
  \item $T(c_{1}v_{1} + c_{2}v_{2} + \ldots + c_{n}v_{n}) = c_{1}T(v_{1}) + c_{2}T(v_{2}) + \ldots + c_{n}T(v_{n})$
  \item $T(\vec{0}) = \vec{0}$
  \item $\image T$ og $\ker T$ er vektorrom (0 er altså alltid med).
  \item $\dim \image T + \dim \ker T = \dim V$ (Se Rank-Nullity --
    Seksjon~\ref{sec:rank-nullity}).
  \item $\dim V < \dim W \Rightarrow$ $T$ kan ikke være surjektiv. (Se også lenger oppe)
  \item $\dim V > \dim W \Rightarrow$ $T$ kan ikke være injektiv. (Se også lenger oppe)
\end{itemize}

\subsection{Eksempel}
Derivasjon er en lineærtransformasjon fra vektorrommet av uendelig deriverbare
funksjoner til seg selv

En $m \times n$-matrise $A$ kan ses som en lineærtransformasjon $T:\R^{n} \rightarrow \R^{m}$:
\[T(v) = A v\]

\section{Matriser}
Matriser er tabeller med høyde og bredde og tall. En $m \times n$-matrise har høyde
$m$ og bredde $n$. Hvis to matriser har samme dimensjoner kan de adderes ved å
addere plassene enkeltvis. Man kan også gange en matrise med en skalar ved å
gange inn tallet i alle ruter.
\[4\imat{1 & 3 \\ 4 & 7 \\ 0 & 2} + \imat{2 & 5 \\ 3 & 8 \\ 2 & -8} = \imat{4 & 12 \\ 16 & 28 \\ 0 & 8} + \imat{2 & 5 \\ 3 & 8 \\ 2 & -8} = \imat{6 & 17 \\ 19 & 36 \\ 2 & 0}\]
%
Vi kan også gange en $m \times n$-matrise med en $n \times o$-matrise, og resultatet blir
en $m \times o$-matrise. Vi gjør dot-produkt mellom rad og kolonne for å resultatet
i hver rute.

Vi kan se på vektorer som matriser der én av langdene er 1. Vi bruker ofte
kolonnevektorer, altså $n \times 1$-matriser. Disse kan ganges (bakfra) med en
$m \times n$-matrise, og resulatet blir en $m \times 1$-vektor. Eksempel:
\[\imat{2 & -3 \\ 1 & 3 \\ -2 & 3}\imat{4 \\ 5} = \imat{2\cdot4 -3\cdot5 \\ 1\cdot4 + 3\cdot5 \\ -2\cdot4 + 3\cdot5} = \imat{-7 \\ 19 \\ 7}\]

\subsection{Regneregler}
La $A$, $B$ og $C$ være matriser. La $I$ være identitetsmatrisen, og $O$ være
nullmatrisen. La $a$ og $b$ være skalarer. La størrelsene på matrisene være slik
at operasjonene er definerte. Da holder følgende formler.
\begin{align*}
  A + B &= B + A & \text{+ er kommutativ} \\
  A + (B + C) &= (A + B) + C & \text{+ er assosiativ} \\
  A + O &= O + A = A & \text{$O$ er + sitt identitetselement} \\
  A + -A &= 0 & \text{+ har inverser} \\
  A(BC) &= (AB)C & \text{* er assosiativ} \\
  AI &= IA = A & \text{$I$ er * sitt identitetselement} \\
  A(B + C) &= AB + AC & \text{* distribuerer over +} \\
  (B + C)A &= BA + CA \\
  a(B + C) &= aB + aC \\
  (a + b)C &= aC + aC \\
  a(bC) &= (ab)C \\
  a(BC) &= (aB)C = B(aC)
\end{align*}

\subsection{Ligningssett: $Ac = v$}
For å løse ligningssett kan vi putte ligningene inn i en matrise $A$, og variablene
inn i en kolonnevektor $c$. Det ønskede resultatet fra hver ligning puttes in en
kolonnevektor $v$. Ligningen blir da $Ac = v$. Eksempel:
\begin{align*}
  \begin{array}{cccc}
    4x &- 3y &+ 2z &= 8 \\
    2x &- 4y &- 3z &= -5 \\
    3x &+ y &- z &= 10
  \end{array}
       &\Longrightarrow
  \begin{bmatrix}
    4 & -3 & 2\\
    2 & -4 & -3\\
    3 & 1 & -1\\
  \end{bmatrix}
  \begin{bmatrix}
    x \\ y \\ z
  \end{bmatrix}
       =
  \begin{bmatrix}
    8 \\ -5 \\ 10
  \end{bmatrix}
\end{align*}
%
Dette kan løses ved hjelp av radoperasjoner og Gauss-eliminasjon

\subsection{Radoperasjoner, Gauss-eliminasjon og pivot}
Vi kan skrive opp ligningssettet fra forrige oppgave på følgende måte:
\[\mat{ccc|c}{4 & -2 & 2 & 8 \\ 2 & -4 & -3 & -5 \\ 3 & 1 & -1 & 10}\]
Vi har nå tre radoperasjoner vi kan gjøre:
\begin{itemize}
  \item Gange en rad med et tall $c \ne 0$
  \item Legge til en multippel av én rad til en annen rad
  \item Bytte plass på to rader
\end{itemize}
Disse operasjonene endrer ikke på løsningene på ligningssettet. Man kan også
gjøre radoperasjoner på matriser uten ``løsning'' (altså tall bak streken), men disse
operasjonere vil endre determinant og kolonnerom.

\subsubsection{Trappeform}
Gauss-eliminasjon betyr å gjøre radoperasjoner til du oppnår trappeform.
Definisjonen på trappeform er at alle rader starter med $n$ 0ere etterfulgt av
en 1, og at alle rader har $n$ strengt større enn raden før. Dette holder på
helt til du har rader med bare 0, som du kan ha så mange du vil av, så lenge de
er på bunnen. Hver rad som ikke er bare 0 har som sagt en 1er som første ikke-0,
og den kalles \textbf{pivot-element}. Antall pivot-elementer = $\rank$ til
matrisen.

\textbf{Eksempler}
\begin{align*}
  \begin{bmatrix}
    1 & 2 & 0 \\
    0 & 1 & -4 \\
    0 & 0 & 1
  \end{bmatrix}
  &,
  \begin{bmatrix}
    1 & 0 & -3 \\
    0 & 0 & 1 \\
    0 & 0 & 0
  \end{bmatrix}
  &,
  \begin{bmatrix}
    0 & 1 & -3 \\
    0 & 0 & 1 \\
    0 & 0 & 0 \\
    0 & 0 & 0
  \end{bmatrix}
  &,
  \mat{ccc|c}{
    0 & 1 & 3 & 8 \\
    0 & 0 & 1 & 2 
  }
\end{align*}
%
\textbf{Moteksempler}
\begin{align*}
  \begin{bmatrix}
    1 & 2 & 0 \\
    0 & 1 & -4 \\
    0 & 1 & 1
  \end{bmatrix}
  &,
  \begin{bmatrix}
    1 & 0 & -3 \\
    0 & 0 & 0 \\
    0 & 0 & 1
  \end{bmatrix}
  &,
  \begin{bmatrix}
    2 & 1 & -3 \\
    0 & 1 & 1 \\
    0 & 0 & 0 \\
    0 & 0 & 0
  \end{bmatrix}
  &,
  \mat{ccc|c}{
    0 & 0 & 1 & 2 \\
    0 & 1 & 3 & 8
  }
\end{align*}

\subsubsection{Redusert trappeform}
Ofte er det greit å gjøre radoperasjoner vidre til redusert trappeform. Det
betyr i tillegg til trappeform at hver kolonne med pivot-element ellers bare har
0.

\textbf{Eksempler}
\begin{align*}
  \begin{bmatrix}
    1 & 0 & 0 \\
    0 & 1 & 0 \\
    0 & 0 & 1
  \end{bmatrix}
  &,
  \begin{bmatrix}
    1 & 0 & 2 \\
    0 & 1 & -4 \\
    0 & 0 & 0
  \end{bmatrix}
\end{align*}

\subsubsection{Løse ligningssettet}
Først reduser matrisen til redusert trappeform. Hvis du har noen 0-rader med
ikke-0 løsning, er ligningssettet inkonsistent, og har ingen løsning. (Det
finnes måter å finne ``nærmeste'' løsning, se seksjon~\ref{sec:minste_kvadraters_metode}).
\begin{align*}
  \mat{cc|c}{
  1 & 0 & 4 \\
  0 & 1 & 2 \\
  0 & 0 & 3
          }
          &\Rightarrow \text{Ingen løsninger}
\end{align*}
%
Hvis ligningssettet har et pivot-element i hver kolonne, er det
nøyaktig én løsning
\begin{align*}
  \mat{ccc|c}{
  1 & 0 & 0 & 4 \\
  0 & 1 & 0 & 2 \\
  0 & 0 & 1 & -2 \\
  0 & 0 & 0 & 0
            }
          &\Rightarrow \mathbb{L} = \mat{c}{4 \\ 2 \\ -2}
\end{align*}
%
Hvis ikke alle kolonner har pivot-element, er det uendelig mange løsninger, og
vi parametriserer. Alle kolonner uten pivot-element blir frie variabler
(parameter).
\begin{align*}
  \mat{ccc|c}{
  1 & 0 & -4 & 4 \\
  0 & 1 & 1 & 2 \\
  0 & 0 & 0 & 0 \\
            }
    &\Rightarrow
      \begin{array}{c}
        \text{Introduserer parameter $s$} \\
        x -4s = 4 \\
        y + s = 2 \\
        z = s
      \end{array}
  &\Rightarrow
      \begin{array}{c}
        \text{Løser ut} \\
        x = 4+4s \\
        y = 2-s \\
        z = s
      \end{array}
  &\Rightarrow\mathbb{L} = \mat{c}{4\\2\\0} + s\mat{c}{4\\-1\\1}
\end{align*}

\subsection{Triangulære matriser}
Vi definerer øvre triangulære og nedre triangulære matriser som kvadratiske
matriser der henholdsvis alt under eller alt over diagonalen er 0. Merk at
trappeform i kvadratisk matrise impliserer øvre triangulær. Denne implikasjonen
går ikke begge veier, siden trappeform krever at pivotelementer er 1.

\subsection{Null-, rad- og kolonnerom}
Nullrommet til en matrise er alle vektorer som ganget med matrisen blir
0-vektoren. Dette er da kjernen ($\ker$) til lineærtransformasjon gitt av $A_{m \times n}$.
\[\Null A_{m \times n} = \{v \in \R^{n} | A_{m \times n} v = \nullmat\}\]
%
Kolonnerommet er spennet av vektorene som utgjør kolonnene i matrisen. Dette vil
også være bildet ($\image$) av lineærtransformasjonen.
\[\Col A_{m \times n} = \Sp\{c_{1}, c_{2}, \ldots, c_{n}\}\]
%
Radrommet er spennet av vektorene som utgjør radene i matrisen.
\[\Row A_{m \times n} = \Sp\{r_{1}, r_{2}, \ldots, r_{m}\}\]
Et viktig resultat er at radrommet er ortogonalt komplement til nullrommet. Se
seksjon~\ref{sec:ortogonalt_komplement}.

\subsubsection{Basis for Col or Row}
Gjør gauss-eliminasjon på $A$ til du får trappeform.\\
Radene som ikke er 0 utgjør er en basis for $\Row A$.\\
For en basis for $\Col A$ plukker du ut de kolonnene som inneholder
pivot-elementer, men fra $A$, altså fra matrisen før du gjør radopperasjoner.

\subsection{Determinant}
TODO\@: Hvordan regne determinant, hva det betyr. Kun for kvadratiske mariser.

\[\det(A) = 0 \iff A \text{ er ikke-inverterbar}\]
Se seksjon~\ref{sec:ikke-inverterbare_matriser} for betydningen av dette.

Siden determinanten sier hvordan volumet endrer seg etter en
lineærtransformasjon, og matrisemultiplikasjon gir en matrise som gjør begge
lineærtransformasjonene, gir følgende regel mening
\[\det(AB) = \det(A)\det(B)\]
Dette forteller oss f.eks.\ at et produkt med en ikke-inverterbar matrise alltid
vil være ikke-inverterbart.

\subsection{Rank}
Rank til en $m \times n$-matrise $A$ kan defineres på flere, identiske måter.
\[\rank A = \dim \Col A = \dim \Row A\]
Rank er altså antallet lineært uavhenige kolonnevektorer, og sier oss hvor mange
dimensjoner vi får som ``output'' fra matrisen. Vi kan f.eks.\ finne rank ved å
gjøre gauss-eliminasjon og telle antall pivot-elementer. Det er også verdt å merke seg at
siden $\rank A = \rank A^{\top}$, så er det mulig å regne ut rank den veien man
selv vil.

\subsubsection{Rank-Nullity}\label{sec:rank-nullity}
Et viktig teorem sier at for enhver matrise $A_{m \times n}$ vil følgende holde:
\[\rank A + \dim \Null A = n\]
Rank sier hvor mange dimensjoner vi har ``beholdt'' i bildet, og dimensjonen til nullrommet sier
hvor mange dimensjoner vi har ``mistet'' til 0. Disse to i sum skal være lik
dimensjonen til input, $n$.

\subsection{Transponert}
Den transponerte av en $m \times n$-matrise $A$ er en $n \times m$-matrise $A^{\top}$ der
rader og kolonner har byttet om. Rad 1 er blitt kolonne 1 osv.
\[A = \mat{cc}{3 & 4 \\ 2 & 8 \\ -2 & 7} \hspace{15mm} A^{\top} = \mat{ccc}{3 & 2 & -2 \\ 4 & 8 & 7}\]
Dette betyr at radrom har blitt kolonnerom og omvendt. Determinant og rank har
ikke endret seg.

\subsubsection{Adjunkt}
Når vi har komplekse matriser brukes ofte den adjunkte
$A^{*} = \overline{A^{\top}}$. Dette betyr at matrisen er transponert og
komplekskonjugert. Alle komplekse tall blir ersattet med konjugatene sine ($a+bi
\Rightarrow a-bi$). For reelle matriser vil altså $A^{*} = A^{\top}$. Ofte vil teoremer om
reelle matrisers transponerte være overførbare til komplekse matrisers adjunkte.
Hvis en reell matrise $A = A^{\top}$ kalles den symmetrisk. Hvis en matrise
$A = A^{*}$ kalles den hermitsk. Se seksjon~\ref{sec:symmetriske_matriser}.
\subsubsection{Resultater}
\begin{itemize}
  \item La $A$ være en $m \times n$-matrise, og $\vx \in \C_{n}$, $\vy \in \C_{m}$ være vektorer.
    Da gjelder følgende:
    \[\inner{A\vx,\vy} = \inner{\vx,A^{*}\vy}\]
  \item ${(A+B)}^{\top} = A^{\top} + B^{\top}$
  \item ${(AB)}^{\top} = B^{\top}A^{\top}$
  \item $\rank A = \rank A^{\top}$
  \item $\det A = \det A^{\top}$
  \item $A$ og $A^{\top}$ har de samme egenverdiene, men ikke
    nødvendigvis de samme egenvektorene. Beviset bruker determinanten, og at
    $I_{n} = I_{n}^{\top}$.
    \[\det(A^{\top}-\lambda I_{n}) = \det({(A-\lambda I_{n})}^{\top}) = \det(A-\lambda I_{n})\]
  \item $A^{*}A$ er alltid en hermitsk matrise
  \item $A$ er inverterbar $\iff$ $A^{\top}A$ er inverterbar (hint: determinant av produkt).
\end{itemize}

\subsection{Inverser}
Når det er snakk om inversen til en matrise er det nesten alltid snakk om
kvadratiske matriser.

\subsubsection{Ikke-inverterbare matriser}\label{sec:ikke-inverterbare_matriser}
Følgende er ekvivalent for en kvadratisk $n \times n$-matrise:
\begin{itemize}
  \item $\det A = 0$
  \item $\rank A < n$
  \item $A$ er ikke inverterbar
  \item $\Null A \neq \{0\}$
  \item $A$ har $0$ som egenverdi
  \item Lineærtransformasjonen $T(x) = Ax$ er ikke-injektiv
\end{itemize}

\subsubsection{Kvadratiske matriser}
En $n \times n$-matrise $A$ sies å være inverterbar hvis $\rank A = n$, altså at alle
kolonnene er lineært uavhengige. Dette betyr at dimensjonen til bildet er lik
dimensjonen til input, som igjen betyr at nullrommet er 0-dimensjonalt. Dette
betyr igjen at transformasjonen er bijektiv, som betyr inverterbar. Vi kaller
inversen $A^{-1}$. Den er definert slik:

\[A \cdot A^{-1} = A^{-1} \cdot A = I_{n}\]
%
Det finnes kun én mulig verdi for $A^{-1}$, og den vil funke på begge sider av
$A$. Vi kan regne ut $A^{-1}$ ved å bruke radoperasjoner. Vi setter inn $A$ på
venstre side av streken, og identitetsmatrisen på andre side av streken.
Deretter gjør vi radoperasjoner til venstresiden er blitt identitetsmatrisen. Da
vil høyresiden være inversen til $A$.

\begin{align*}
  \begin{bmatrix}
    1 & 3 & \vline & 1 & 0 \\
    2 & 7 & \vline & 0 & 1
  \end{bmatrix}
      \sim
  \begin{bmatrix}
    1 & 3 & \vline &  1 & 0 \\
    0 & 1 & \vline & -2 & 1
  \end{bmatrix}
      \sim
  \begin{bmatrix}
    1 & 0 & \vline &  7 & -3 \\
    0 & 1 & \vline & -2 & 1
  \end{bmatrix}
\end{align*}
%
Inversen til $\imat{1 & 3 \\ 2 & 7}$ er altså $\imat{7 & -3 \\ -2 & 1}$

\subsubsection{Left-inverser}
Gitt en $m \times n$-matrise $A$ der $\rank A = n$ (lineært uavhengige kolonner), er
en left-invers $A_{left}^{-1}$ en $n \times m$-matrise slik at
$A_{left}^{-1} \cdot A = I_{m}$. For å finne en left-invers kan vi bruke
$A^{\top} \cdot A$, som er en inverterbar $n \times n$-matrise.
\begin{align*}
  {(A^{\top}A)}^{-1}(A^{\top}A) &= I_{m} \\
  ({(A^{\top}A)}^{-1}A^{\top}) \cdot A &= I_{m} \\
  {(A^{\top}A)}^{-1} A^{\top} &= A_{left}^{-1}
\end{align*}

\section{Indreprodukt og ortogonalitet}\label{sec:indreprodukt}
Indreproduktet er en slags generalisering av dotproduktet som funker også i
merkeligere vektorrom. Vi kan definere et indreprodukt $\inner{\vv,\vw} \in \C$
mellom to vektorer i vektorrommet $V$. Operasjonen må tilfredsstille tre krav:
\begin{itemize}
\item \textbf{Symmetri} $\inner{\vv,\vw} = \overline{\inner{\vw,\vv}}$
\item \textbf{Positivitet} $\inner{\vv,\vv} \geq 0$, og $\inner{\vv,\vv} = 0$ kun
  hvis $\vv = 0$
\item \textbf{Linearitet} $\inner{\vv, (a\vw + b\vu)} = a\inner{\vv,\vw} + b\inner{\vv,\vu}$
\end{itemize}
Et vektorrom med et definert indreprodukt kalles et indreproduktrom. Merk at
Wikipedia har definert indreproduktet som lineært i første ledd, mens Matte
3-forelesningsnotatene definerer i andre ledd.

\subsection{Resultater}
\begin{itemize}
  \item $\inner{\vv,c \vw} = c\inner{\vv,\vw} \Longrightarrow \inner{c \vv,\vw} = \overline{c}\inner{\vv,\vw}$
  \item $\inner{\vu + \vv,\vw} = \inner{\vu,\vw} + \inner{\vv,\vw}$
  \item $\inner{\vu , \vv+\vw} = \inner{\vu,\vv} + \inner{\vu,\vw}$
  \item $\inner{0, \vv} = 0$

\end{itemize}

\subsection{Lengde og vinkel}
Vi definer lengden av $\vv$, $||\textbf{v}|| = \sqrt{\inner{\vv,\vv}}$. På
grunn av positivitetskravet vet vi $||\vv|| \in \R$ og $||\vv||=0 \iff \vv=0$.
Lengden mellom to vektorer $\vv$ og $\vu$ blir da lik $||\vv - \vu||$

Vi definerer også vinkelen mellom to vektorer ved å bruke
\[Re \inner{\vv,\vw} = ||\vv|| ||\vw|| \cos \theta\]
og velge en vinkel $\theta \in [0,\pi]$. Vi vet at
$-1 \leq \cos \theta \leq 1$, så vinkelen vil alltid være definert. Det er ikke
alle som gidder å definere vinkler mellom vekoter i vektorrom med komplekse
indreprodukt.

\subsection{Ortogonalitet}
Vi definerer ortogonalitet på følgende måte
\[\inner{\vv,\vw} = 0\]

Vi kan også si at en vektor $\vv \in V$ står ortogonalt på et underrom
$U \subseteq V$. Dette vil si at $\inner{\vv,\vu} = 0$ for alle $\vu \in U$. Vi får f.eks.\@
slike underrom hvis vi tar spennet over ortogonale vektorer til $\vv$. Utifra
definisjonen på indreprodukt:
\begin{align*}
  \inner{\vv,\vu} = 0 &\land \inner{\vv,\vw} = 0 \\
  a \inner{\vv,\vu} = 0 &\land b \inner{\vv,\vu} = 0 \\
  \inner{\vv,a \vu} = 0 &\land \inner{\vv,b \vu} = 0 \\
  \Longrightarrow \inner{\vv, a\vu + b\vw} &= 0+0 = 0
\end{align*}

\subsubsection{Resultater}
\begin{itemize}
  \item Hvis \vv{} og \vw{} er ortogonale, er
    $||\vv + \vw||^{2} = ||\vv||^{2} + ||\vw||^{2}$. ``\textbf{Pytagoras}'' gjelder altså.
\end{itemize}

\subsection{Ortogonalt komplement}\label{sec:ortogonalt_komplement}
Gitt et indreproduktrom $V$ og et underrom $U \subseteq V$, vil det ortogonale
komplementet til $U$, $U^{\bot}$, være alle vektorer som står ortogonalt på
alle vektorer i $U$. Denne mengden utgjør et underrom av $V$, siden den er
lukket under addisjon og skalarmultiplikasjon. $U$ og $U^{\bot}$ er lineært
uavhengige, og tilsammen vil de spenne ut hele $V$.

\subsubsection{Eksempel / Teorem + Bevis}
Gitt en $m \times n$-matrise $A$ vil $\Row A$ være ortogonalt komplement til
$\Null A$. Vi setter først opp ligningen for nullromet.
\begin{align*}
  \mat{cccc}{a_{1,1} & a_{1,2} & \cdots & a_{1,n} \\ a_{2,1} & a_{2,2} & \cdots & a_{2,n} \\ \vdots & \vdots & \ddots & \vdots \\ a_{m,1} & a_{m,2} & \cdots & a_{m,n}}\mat{c}{x_{1} \\ x_{2} \\ \vdots \\ x_{n}} = \mat{c}{0 \\ 0 \\ \vdots \\ 0}
\end{align*}
Vi kaller løsningsvektoren $\vx$, og vet at den er i nullrommet til $A$.
Husk at matrisemultiplikasjon kan deles opp som indreprodukt mellom rader og
kolonner. Vi nummererer radene $r_{i}$ for $1 \le i \le m$. Ligningen forteller oss
altså at
\[\inner{r_{i}, \vx} = 0, \hspace{20mm} 1 \le i \le m\]
Dette betyr at enhver rad i $A$ står ortogonalt på $\vx$, altså enhver vektor i
nullrommet. Dette medfører også at envher lineærkombinasjon av radvektorer vil
stå ortogonalt på nullrommet. Vi vet altså nullrommet og radrommet er
ortogonale, men for at $\Null A$ og $\Row A$ skal være hverandres
ortogonale komplement må de til sammen spenne ut hele $\R_{n}$. Dette viser vi
med \textit{rank-nullity}. Se seksjon~\ref{sec:rank-nullity}.
\begin{align*}
  \dim \Sp\{\Row A, \Null A\} &= \dim Row A + \dim \Null A \\
                              &= \rank A + \dim \Null A \\
                              &= n
\end{align*}
Dette tilsammen gir oss at
\[{(\Row A)}^{\bot} = \Null A \hspace{10mm} {(\Null A)}^{\bot} = \Row A\]

\subsubsection{Resultater}
\begin{itemize}
  \item $\Sp\{U, U^{\perp}\} = V$
  \item $\dim U + \dim U^{\bot} = \dim V$
  \item $U^{\bot^\bot} = U$
  \item ${(\Row A)}^{\bot} = \Null A$ (Se bevis over)
\end{itemize}

\subsection{Eksempler}
I $R^{n}$ defineres ofte indreproduktet til å være dotproduktet. Med
matrisemultiplikasjon skrives det slik:
\[\inner{\vv,\vw} = \vv^{\top}\vw\]
I $\C^{n}$ kan vi definere indreproduktet lignende, men med adjunkt istedet:
\[\inner{\vv,\vw} = \vv^{*}\vw = \overline{\vv}^{\top}\vw\]
I vektorrommet av kontinuerlige funksjoner på intervallet $[a,b]$ kan vi
definere indreprodukt som
\[\inner{\vv,\vw} = \int_{a}^{b} \vv(x)\vw(x) dx\]

\section{Projeksjon}\label{sec:projeksjoner}
Gitt en vektor $\vw$ i indreproduktrommet $V$ og et underrom $U$, er $P_{U}(\vw)$ vektoren i $U$
nærmest $\vw$. Vi vet da at vektoren fra $P_{U}(\vw)$ til $\vw$ står ortogonalt på
underrommet $U$. Lengden på denne vektoren er avstanden. Ortogonalitet og lengde
defineres med indreproduktet i $V$.

\subsection{Projeksjon til linje}
Vi definerer først $P_{\vv}(\vw)$ der $\vv$ og $\vw$ er vektorer i $V$.
\[P_{\vv}(\vw) = \frac{\inner{\vv,\vw}}{\inner{\vv,\vv}}\vv\]
Resultatet er en vektor i $\Sp\{\vv\}$, altså på linjen til $\vv$. Vektoren
fra $P_{\vv}(\vw)$ til $\vw$ kan vil kalle $\vx = \vw - P_{\vv}(\vw)$. Lengden
$||\vx||$ er da den korteste avstanden fra $\Sp{\vv}$ til $\vw$, og $\vx$ står
ortogonalt på $\vv$, altså:
\[\vw-P_{\vv}(\vw) \perp \vv  \iff \inner{\vw-P_{\vv}(\vw), \vv} = 0\]

\subsection{Ortogonal basis}
For å gjøre projeksjon til et underrom $U$, er det kjekt å ha en ortogonal
basis for $U$. $\B=(\vu_{1}, \vu_{2}, \ldots, \vu_{n})$ slik at
$\inner{\vu_{i}, \vu_{j}} = 0 for alle 1 \leq i,j \leq n, i \neq j$. Dette har vi en
algoritme for å gjøre.

\subsubsection{Gram-Schmidt}
La $\B=(\vv_{1}, \vv_{2}, \ldots, \vv_{n})$ være en basis for underrommet $U$ i
et indreproduktrom. Vi ønsker å finne en ortogonal basis
$\B_{\perp}=(\vu_{1}, \vu_{2}, \ldots, \vu_{n})$.

\begin{align*}
  \vu_{1} &= \vv_{1} \\
  \vu_{2} &= \vv_{2} - P_{\vu_{1}}(\vv_{2}) \\
  \vu_{3} &= \vv_{3} - P_{\vu_{1}}(\vv_{3}) - P_{\vu_{2}}(\vv_{3}) \\
          &\vdots \\
  \vu_{n} &= \vv_{n} - P_{\vu_{1}}(\vv_{n}) - \cdots - P_{\vu_{n}}(\vv_{n})
\end{align*}

\subsection{Projeksjon til underrom}
Når vi skal projisere $\vv$ til et underrom $U$, bruker vi en ortogonal basis
for $U$.
\[B_{\perp} = (\vu_{1}, \vu_{2}, \ldots, \vu_{n})\]
Projeksjonen av $\vv$ blir da lik summen av projeksjonene ned på hver vektor i
den ortogonale basisen.
\[P_{U}(\vv) = P_{\vu_{1}}(\vv) + P_{\vu_{2}}(\vv) + \ldots + P_{\vu_{n}}(\vv)\]

\subsection{Resultater}
\begin{itemize}
\item $P_{U}(\vv)$ er en lineærtransformasjon:
\[P_{U}(a\vv + b\vw) = aP_{U}(\vv) + bP_{U}(\vw)\]
\item $\vv-P_{U}(\vv)$ står ortogonalt på $U$
\end{itemize}

\section{Egenverdier og egenvektorer}
La $T:V\rightarrow V$ være en lineærtransformasjon. Da er skalaren $\lambda$ en
egenverdi hvis det finnes en ikke-triviell vektor $v \in V$ slik at
\[T(v) = \lambda v\]
$v$ kalles egenvektor for $T$ med egenverdi $\lambda$.
Merk at $\lambda$ kan være 0 $\Longrightarrow$ $v \in \ker T\setminus\{0\}$ er en egenvektor for
$T$ med $\lambda = 0$.

\subsection{Egenrom}
Gitt to egenvektorer $\vv$ og $\vu$ med samme egenverdi $\lambda$, vet vi at en
lineærkombinasjon av dem også vil være en egenvektor med egenverdi $\lambda$. Bevis:
\[T(a \cdot \vv + b \cdot \vu) = a \cdot T(\vv) + b \cdot T(\vu) = a \cdot \lambda \vv + b \cdot \lambda \vu = \lambda(a \cdot \vv + b \cdot \vw)\]
Dette betyr at alle $\vu \in \Sp\{\vv_{1}, \vv_{2}, \cdots, \vv_{n}\}$ er egenvektorer med
egenverdi $\lambda$, gitt at alle $\vv_{n}$ er egenvektorer med egenverdi $\lambda$.
Alle egenvektorer med en gitt egenverdi $\lambda$ utgjør derfor et underrom, som
kalles egenrommet for $\lambda$. Husk at 0-vektoren ikke er en egenvektor, selv om den
``er med'' i alle underrom, også egenrom.

\subsection{Å finne egenverdier og -rom}
Gitt en kvadratisk $n \times n$-matrise $A$ ønsker vi å finne en ikke-triviell vektor $\vv \in \R^{n}$
slik at
\begin{align}
  A\vv = \lambda \vv \\
  A\vv - \lambda I_{n} \vv = \nullvec \\
  (A - \lambda I_{n})\vv = \nullmat \label{eq:ai}
\end{align}
Ligning~(\ref{eq:ai}) har kun ikke-trivielle løsninger hvis kolonnene ikke er
lineært uavhengige, altså
\[\det (A-\lambda I_{n}) = 0\]
Dette gir et polynom i $\lambda$ av grad $n$. Røttene til polynomet blir da
egenverdier. Når vi har bestemt oss for en egenverdi $\lambda$ kan vi finne egenrommet til
den ved å løse ut for
\begin{align*}
  A\vx &= \lambda \vx \\
  (A - \lambda I_{n})\vx &= 0
\end{align*}
Siden determinanten til venstresiden er 0, vil den ha uendelig mange løsninger.
Vi må derfor parameterisere.

\subsubsection{Eksempel}
TODO

\subsection{Antall egenverdier}
Egenverdier vil være røtter i polynomet vi fant. Antallet ganger en egenverdi
$\lambda$ forekommer som rot, kalles den \textbf{algebraiske multiplisiteten} til $\lambda$.
For eksempel i polynomet
\[{(\lambda - 4)}^{2}(\lambda + 2) = 0\]
vil $\lambda=4$ ha en algebraisk multiplisitet på $2$, og $\lambda=-2$ ha en algebraisk
multiplisitet på $1$.

Algebraens fundamentalteorem forteller oss at summen av algebraiske
multiplisiteter alltid er $n$, altså dimensjonen på matrisen, men ikke alle
røttene til polynomet er nødvendigvis reelle tall.

I tillegg til algebraisk multiplisitet har alle egenverdier også
\textbf{geometrisk~multiplisitet}. Det er definert som dimensjonen på egenrommet
til $\lambda$. Regel for enhver egenverdi $\lambda$:
\[1 \leq \text{geometrisk multiplisitet} \le \text{algebraisk multiplisitet}\]
Vi sier at egenverdien er defekt hvis den geometriske multiplisiteten er lavere
enn den algebraiske multiplisiteten. Dette kan også skje med komplekse
egenverier.

Hvis ingen egenverdier er defekte vil summen av geometriske
multiplisiteter være lik dimensjonen på rommet. Egenvektorer med ulik
egenverdi er alltid lineært uavhengige. Det betyr at man har nok egenvektorer
til å lage en basis for rommet av kun egenverdier. Dette er kravet for
diagonalisering.

\subsection{Komplekse egenverdier}
I reelle matriser gjelder følgende: Hvis $\lambda$ er en kompleks egenverdi, er
også $\overline{\lambda}$ en kompleks egenverdi (anslagsvis med samme
algebraiske multiplisitet). Dette betyr at en reell matrise $A_{n \times n}$ der
$n$ er et oddetall, må ha minst én reell egenverdi.

\subsection{Resultater}
\begin{itemize}
  \item Hvis $\lambda_{1}, \lambda_{2}, \ldots \lambda_{n}$ er ulike
    egenverdier for $T:V\rightarrow V$, med henholdsvise egenvektorer
    $v_{1}, v_{2}, \ldots, v_{n}$, vil vektorene være lineært uavhengige.
  \item Hvis $v_{1}, v_{2}, \ldots, v_{n}$ er egenvektorer i $V$ med samme egenverdi
    $\lambda$, er $\forall u \in \Sp{v_{1}, v_{2}, \ldots, v_{n}}\setminus\{0\}$ egenvektorer med
    egenverdi $\lambda$
  \item Hvis matrisen $A$ har egenverdi $\lambda$, vil $A^{n}$ ha egenverdi $\lambda^{n}$
  \item Å ha $0$ som egenverdi $\iff$ $\ker T$ ikke er triviell $\iff$ $\lbrack T \rbrack$ er
    ikke-inverterbar.
  \item For en $n \times n$-matrise vil $1 \le \dim \Sp\{alle egenvektorer\} \le n$
\end{itemize}

\section{Diagonalisering}
For en $n \times n$-matrise $A$ med n lineært uavhengige egenvektorer (sum av geometrisk
multiplisitet = $n$), kan vi lage $n \times n$-matriser $P$ og $D$ slik at
\[ A = PDP^{-1} \]
der $D$ er en diagonalmatrise, altså $D_{ij} = 0, i \ne j$. Dette gjør det lett å
eksponensiere $A$ siden
\begin{align*}
  A^{x} &= {(PDP^{-1})}^{x} \\
        &= PD P^{-1} \cdot P D P^{-1} \cdot \ldots \cdot P DP^{-1} \\
        &= PD (P^{-1} \cdot P) D (P^{-1} \cdot \ldots \cdot P) DP^{-1} \\
        &= PD \cdot I_{n} \cdot D \cdot I_{n} \cdot \ldots \cdot DP^{-1} \\
        &= PD^{x}P^{-1}
\end{align*}
og vi lett kan regne ut $D^{x}$ siden $D$ er en diagonalmatrise, og hvert tall i
matrisen dermed kan eksponensieres for seg.

\textbf{Merk.} En matrise kan være diagonaliserbar selv om den ikke er
inverterbar (å ha 0 som egenverdi går fint).

\subsection{Hvordan komme fram til P og D}
Begynn med n lineært uavhengige egenvektorer til $A$: $v_{1}, v_{2}, \cdots, v_{n}$.
Når $A$ ganges med $v_{x}$ som kolonnevektor, blir svaret $\lambda_{x} v_{x}$. La så
vektorene være kolonner i matrisen $P$:
\begin{align*}
  P &= \begin{bmatrix}
    v_{1} & v_{2} & \cdots & v_{n}
  \end{bmatrix} \\
  AP &= \begin{bmatrix}
    \lambda_{1}v_{1} & \lambda_{2}v_{2} & \cdots & \lambda_{n}v_{n}
  \end{bmatrix}
\end{align*}
La så $D$ være en diagonalmatrise med $\lambda_{1}, \lambda_{2}, \cdots, \lambda_{n}$ som diagonaler.
\begin{align*}
  D &= \begin{bmatrix}
    \lambda_{1} & 0 & \cdots & 0 \\
    0 & \lambda_{2} & \cdots & 0 \\
    \vdots & \vdots & \ddots & 0 \\
    0 & 0 & 0 & \lambda_{n}
  \end{bmatrix} \\
  PD &= \begin{bmatrix}
    \lambda_{1}v_{1} & \lambda_{2}v_{2} & \cdots & \lambda_{n}v_{n}
  \end{bmatrix}
\end{align*}
Dette gir det totale uttrykket
\[AP = PD\]
Siden $P$ består av lineært uavhengige kolonnevektorer er den inverterbar. Vi
ganger med $P^{-1}$ på begge sider og får
\[A = PDP^{-1}\]

\subsection{Komplekse matriser}
En reell $n \times n$-matrise er realt diagonaliserbar hvis og bare hvis det finnes
en basis for $\R_{n}$ utgjort av egenvektorer til matrisen. \\
En kompleks $n \times n$-matrise er diagonaliserbar hvis og bare hvis det finnes en
basis for $\C_{n}$ utgjort av egenvektorer til matrisen. Husk at den geometriske
multiplisiteten til et egenrom ikke alltid er like stor som den algebraiske,
uavhengig om du tillater komplekse tall.

\subsection{Diagonaliserbare lineærtransformasjoner}
La $T:V\rightarrow V$ være en lineærtransformasjon. Vi sier at $T$ er diagonaliserbar hvis
man kan lage en basis for $V$ av egenvektorer $v_{1}, v_{2}, \ldots, v_{n}$ til $T$.
Matrisen som beskriver $T$ med hensyn på egenvektorbasisen er en diagonalmatrise
$D$ med diagonal $\lambda_{1}, \lambda_{2}, \ldots, \lambda_{n}$.

\subsection{Resultater}
For enhver $n\times n$-matrise $A$ med $n$ lineært uavhengige egenvektorer finnes
matriser $P$ og $D$ der $D$ er en diagonalmatrise, slik at
\begin{align*}
  A &= PDP^{-1} \\
  P^{-1}AP &= D
\end{align*}

\section{Symmetriske matriser}\label{sec:symmetriske_matriser}
En reell matrise $A$ er symmetrisk dersom $A = A^{\top}$. Det kan vises at
alle symmetriske $n \times n$ matriser har $n$ reelle egenverdier (i geometrisk
multiplisitet), og er dermed diagonaliserbare med reell diagonalmatrise. Vi kan
også vise at egenvektorer med ulik egenverdi alltid vil være ortogonale på
hverandre:

En $n \times n$-matrise som har $n$ ortogonale egenvektorer kalles \textbf{ortogonalt
diagonaliserbar}

\subsection{Hermitske matriser}
For en kompleks $n \times n$-matrise $A$ definerer vi at den er hermisk hvis $A = A^{*}$
der $A^{*} = \overline{A^{\top}}$. Hermitske matriser vil ha reell diagonal, og hvis
hele matrisen er reell er den også symmetrisk. Funnene fra symmetriske matriser
gjelder generelt for hermitske matriser:

En hermitsk $n \times n$-matrise har $n$ reelle egenverdier, og er ortogonalt diagonaliserbar.

\subsection{Resultater}
\begin{itemize}
\item $A\text{ er symmetrisk} \iff A\text{ er reell } \wedge A\text{ er hermitsk}$
\item En $n \times n$-matrise er ortogonalt diagonaliserbar hvis og bare hvis
  den er hermitsk/symmetrisk.
\item Alle hermitske og symmetriske $n \times n$-matriser har $n$ reelle
  egenverdier (talt i geometrisk multiplisitet).
\end{itemize}

\section{Minste kvadraters metode}\label{sec:minste_kvadraters_metode}
Minste kvadraters metode er en måte å finne den ``nærmeste'' løsningen til et
inkonsistent lingsningssett / ligningssett med for mange ligninger.
Ligningssettene skrives på formen $Ax = b$ der $A$ er en $m \times n$-matrise og $b$
er en kolonnevektor av lengde $m$. Løsningen $x$ skal være en kolonnevektor av
lengde $n$. Problemer oppstår når $b$ ikke ligger i kolonnerommet til $A$. Da
finnes det ikke noen $x$ som transformert med $A$ havner i $b$. Dette kan skje når
$A$ har færre enn $m$ uavhengige kolonnevektorer. Da vil kolonnerommet være et strengt
underrom av $\R_{m}$, og $b$ kan dermed ligge utenfor.

\textbf{Merk:} Når vi skal bruke minste kvadraters metode kan vi ikke begyne med
radoperasjoner, siden radoperasjoner endrer kolonnerommet.

Vi definerer avstanden fra $b$ til kolonnerommet som $||b-P_{\Col A}(b)||$,
siden vi vet at projeksjonen er den nærmeste vektoren. Vi ønsker altså å finne
en vektor $\hat{x}$ slik at
\[A\hat{x} = P_{\Col A}(b)\]
Fra Seksjon~\ref{sec:projeksjoner} om projeksjoner vet vi at $b-P_{\Col A}(b)$ står
ortogonalt på $\Col A$. Vi vet også at $\Col A = \Row A^{\top}$. Fra seksjonen om
ortogonale komplementer vet vi at $\Row A$ er ortogonalt komplement til $\Null
A$. Alt dette gir oss at
\begin{align*}
  b-A\hat{x} &\in \Null A^{\top} \\
  A^{\top}(b-A\hat{x}) &= 0 \\
  A^{\top}b &= A^{\top}A\hat{x}
\end{align*}

Dette systemet kan vi løse, og de kalles normalligningene. $A^{T}A$ er en $n\times
n$-matrise, og dersom den er inverterbar finnes det nøyaktig én løsning
$\hat{x}$.

\textbf{Merk.} Det kan finnes flere løsninger for $\hat{x}$, men det er kun étt
punkt i $\Col A$ som er nærmest $b$.

\textbf{Merk.} For komplekse matriser er det den adjunkte $A^{*}$ i stedet for $A^{\top}$.

\subsection{Interpolasjon og regresjon}
Gitt et sett med $n$ punkter
$(x_{1}, y_{1}), (x_{2}, y_{2}), \ldots, (x_{n}, y_{n})$, der $x$ene er unike, kan
vi prøve å lage et polynom $P(x)$ slik at $P(x_{i}) = y_{i}$ for alle $i \in [1,n]$.
Et polynom av grad $m$ har $m+1$ frie variabler. Vi setter inn x-verdier og
ønsket y-verdi som ligningssett på formen.
\[ax^{m} + bx^{m-1} + \cdots + cx + d = y\]
Dersom vi har flere ligninger enn ukjente må vi bruke minste kvadrats metode.

\subsubsection{Eksempel}
TODO

\subsubsection{Resultater}
Det finnes uendelig mange 2.-gradspolynomer som skjærer 2 punkter, men
nøyaktig ett som skjærer 3.

Minste kvadraters metode minimerer ``avstand'' mellom polynomet og punktenes y-verdi.
Avstand defineres etter indreproduktet
\[||\mat{c}{P(x_{1})\\P(x_{2})\\\vdots} - \mat{c}{y_{1}\\y_{2}\\\vdots}|| =
  \sqrt{\Sigma_{i=1}^{n} {(P(x_{i})-y_{i})}^{2}}\]
Dette betyr i praksis at regresjonen minimerer summen av kvadratavvik på y-aksen.

\section{Stokastiske matriser og Markovkjeder}
En sannsynlighetsvektor er er vektor der alle koordinatene er større eller lik
0, og koordinatsummen er 1. En stokastisk matrise er en $n \times n$-kvadratisk
matrise der kolonnene er sannsynlighetsvektorer. Stokastiske matriser kan ses
som sannsynlighetene for tilstandsforandringer mellom alle ordnede par av
tilstander. Kolonnen er nåværende tilstand, og raden er neste tilstand.
\[
  \mat{ccc}{
    P(A_{1} | A_{0}) & P(A_{1} | B_{0}) & P(A_{1} | C_{0}) \\
    P(B_{1} | A_{0}) & P(B_{1} | B_{0}) & P(B_{1} | C_{0}) \\
    P(C_{1} | A_{0}) & P(C_{1} | B_{0}) & P(C_{1} | C_{0}) \\
  }
\]
Produktet av en stokastisk matrise og en sannsynlighetsvektor vil
være sannsynlighetene for hver tilstand i neste ``runde'', igjen en
sannsynlighetsvektor.

Gitt en sannsynlighetsvektor $\textbf{x}_{0}$ og en stokastisk matrise $M$ kan
vi definere \textbf{Markovkjeden}.
\[\{\vx_{0}, \vx_{1}, \vx_{2}, \vx_{3}, \ldots\}, \hspace{10mm} \vx_{n} = M^{n}\vx_{0}\]

\subsection{Likevektsvektor}
En likevektsvektor $\vq$ til en stokastisk matrise $M$ er en
sannsynlighetsvektor som tilfredsstiller
\[M\vq = \vq\]
Altså er likevektsvektoren en egenvektor med egenverdi 1. En stokastisk matrise $M$
vil alltid ha egenverdien 1 (Hint: $\lambda=1$ i $M^{\top}$). Vi finner likeveksvektoren ved å finne en
parameterisering til egenrommet til $\lambda=1$, og sette parametere slik at
koordinatsummen blir $1$.

\subsection{Regulære stokastiske matriser}
En stokastisk matrise $M$ er regulær hvis det finnes en $k \ge 1$ slik at alle
elementer i $M^{k}$ er større enn $0$.
\[\begin{array}{ccc}
  \text{Eksempel} & \hspace{20mm} & \text{Moteksempel} \\
  \mat{cc}{0.8 & 1 \\ 0.2 & 0} & & \mat{cc}{0.8 & 0 \\ 0.2 & 1}
  \end{array}\]
I moteksempelet ser du at kolonne nummer 2 ``fanger'' alt som kommer inn i den.
Det er aldri mulig for noe i tilstand 2 å komme til tilstand 1, så kolonne 2 vil
alltid ha $0$en.
Likevekstvektoren er unik for regulære stokastiske matriser. Det betyr at
markovkjeden konvergerer mot det samme, uansett hvilken sannsynlighetsvektor
$\vx_{0}$ er.

\subsection{Resultater}
Gitt en regulær stokastisk matrise $M$ konvergerer $\lim_{x\to\infty}M^{k}$ mot en
regulær stokastisk matrise $M'$ slik at alle sannsynlighetsvektorer ganget med
$M'$ blir samme vektor. Den ser slik ut, der $q$ er likevektsvektoren (kolonne):
\[
  \mat{cccc}{q & q & \cdots & q}
\]

\section{Differensialligningssett}
Følgende kalles et \textit{Førsteordens lineært og homogent system av
  differensialligninger med konstante koeffisienter}.
\[
\begin{array}{ccccccccc}
  y'_1(t) &=& a_{11}y_1(t) &+& a_{12}y_2(t) &+& \ldots &+& a_{1n}y_n(t) \\
  y'_2(t) &=& a_{21}y_1(t) &+& a_{22}y_2(t) &+& \ldots &+& a_{2n}y_n(t) \\
  \vdots && \vdots && \vdots && \ddots &&  \vdots \\
  y'_n(t) &=& a_{n1}y_1(t) &+& a_{n2}y_2(t) &+& \ldots &+& a_{nn}y_n(t) \\
\end{array}
\]

Det kan også skrives på formen $\vy' = A\vy$ der $\vy$ er en $n$-dimensjonal vektor av
deriverbare funksjoner og $A$ er en $n \times n$-matrise med alle koefisientene.

\begin{align*}
  \vy(t) = \mat{c}{y_1(t) \\ y_2(t) \\ \vdots \\ y_n(t)} &
\hfill &                                                    
A = \mat{cccc}{
a_{11} & a_{12} & \cdots & a_{1n} \\
a_{21} & a_{22} & \cdots & a_{2n} \\
\vdots & \vdots & \ddots & \vdots \\
a_{n1} & a_{n2} & \cdots & a_{nn}
}
\end{align*}

I dette faget forholder vi oss kun til reelle koeffisienter.

La $\vv$ være en egenvektor til $A$ med egenverdi $\lambda$. Da vil følgende holde:
\[(\vv e^{\lambda t})' = \vv\lambda e^{\lambda t} = A\vv e^{\lambda t}\]
$\vv e^{\lambda t}$ er altså en løsning til systemet.

\subsection{Superposisjonsprinsippet}
Gitt to løsninger $\vy'_1 = A\vy_1$ og $\vy'_2 = A\vy_2$

Løsningene på systemet er en $n$-dimensjonalt vektorrom.

\clearpage
\appendixpage{}
\addappheadtotoc{}
\begin{appendices}
\section{Komplekse tall}
Vi definerer $i^{2} = -1$. Dette gir oss $\sqrt{-1} = i$, men vi må være veldig
obs på å bruke kvadratrøtter slik. Husk at $x^{2} = -1$ har to løsninger, $i$ og
$-i$. Hvis vi mikser og trikser for mye med kvadratrøtter får vi fort $1 = -1$.

Gitt komplekse tall $q = a + bi$, og $w = c + di$ har vi følgende regler.
\begin{align*}
  \Rea q &= \Re(q) = a \\
  \Ima q &= \Im(q) = b \\
  q + w &= a+c + (b+d)i \\
  q \cdot w &= (a+bi)(c+di) = ac - bd + (ad+bc)i \\
  \bar{q} &= a - bi\\
  q \cdot \bar{q} &= a^{2} + b^{2} \\
  \frac{q}{w} &= \frac{q\bar{w}}{w\bar{w}} = \frac{q\bar{w}}{c^{2}+d^{2}}
\end{align*}
Vi kan også se på komplekse tall som koordinater, både kartesiske og polare. Å
summere er vanlig vektoraddisjon. Å gange sammen to komplekse tall blir å
summere vinkelene, og multiplisere lengdene. Å gange med $i$ er f.eks. å rotere
$\ang{90}$ mot klokka. $\bar{q}$ er $q$ speilet om den reelle aksen.

\subsection{Polare koordinater}
Man kan definere et komplekst tall ut ifra vinkel til den positive reelle aksen,
og avstand fra origo.
\[re^{i\theta} = r\cos(\theta) + ri\sin(\theta)\]
Gitt to komplekse tall $q = re^{i\theta}$ og $w = se^{i\phi}$ har vi følgende regler:
\begin{align*}
  q \cdot w &= rse^{i(\theta + \phi)} \\
  \frac{q}{w} &= \frac{r}{s}e^{i(\theta-\phi)} \\
  \bar{q} &= re^{-i\theta} \\
  \Rea q &= \Re(q) = r\cos(\theta) \\
  \Ima q &= \Im(q) = r\sin(\theta)
\end{align*}

\subsection{Resultater}
\begin{itemize}
  \item Ligningen $x^{4} = 1$ har 4 løsninger, alle på den komplekse
    enhetssirkelen.
  \item Ethvert polynom av grad $n$ har $n$ røtter (i algrebraisk multiplisitet).
\end{itemize}

\section{Sannsynlighet}
Vi gjør en liten oppfriskning på sannsynlighetsregning og definisjonene som
gjelder der. Vi bruker en 6-sidet terning som eksempel. Terningen har 6 utfall
${1,2,3,4,5,6}$ med lik sansynlighet. Vi definerer hendelsene
\begin{align*}
  A &= \{1,3,5\} \\
  B &= \{4,5,6\} \\
  C &= \{3\} \\
  S &= \{1,2,3,4,5,6\}
\end{align*}
Vi kaller her $S$ for utfallsrommet, som er en spesiell hendelse som alltid
inntreffer (inneholder alle utfall). Vi definerer vidre operatorer på hendelser.
\begin{align*}
  A \cup B &= \{1,3,4,5,6\} &\text{Union (utfall med i minst én)}\\
  A \cap B &= \{5\} &\text{Snitt (utfall med i begge)} \\
  \bar{A} &= \{2,4,6\} &\text{Komplement ($S \setminus A$)}
\end{align*}
Vi lager i tillegg en ekstra definisjon:
\begin{center}
  \textit{To hendelser $A$ og $B$ er disjunkte dersom $A \cap B = \emptyset$}
\end{center}
Vi ser da at $B$ og $C$ er disjunkte, i tillegg til at $A$ og $\bar{A}$ alltid
vil være disjunkte.

Vidre har vi aksiomer for sansynlighetsfunksjonen $P$.
\begin{align}
  &P(S) &= 1 \\
  0 \le &P(A) &\le 1 \\
  &P(A \cup B) &= P(A) + P(B) \label{eq:PSUM}
\end{align}
Ligning (\ref{eq:PSUM}) gjelder \textbf{kun} når $A$ og $B$ er uavhengige hendelser!
Utifra disse kan vi utlede alt annet vi trenger å vite, slik som:
\begin{align*}
  P(\bar{A}) &= 1-P(A) \\
  P(A \cup B) &= P(A) + P(B) - P(A \cap B)
\end{align*}

\subsection{Betinget sannsynlighet}
Vi introduserer syntaksen $P(A|B)$, som betyr sannsynligheten for $A$ gitt at
$B$ er inntruffet. Vi ser da på en andel av utfallsrommet til $B$.
\[P(A|B) = \frac{P(A \cap B)}{P(B)}\]
Dette gir oss trivielle ting slik som $P(A|S) = P(A)$ og $P(A|B)+P(\bar{A}|B)=1$.
Et viktig prinsipp for det vi skal holde på med kalles \textbf{Total sannsynlighet}.
\begin{align*}
  P(A) &= P(A \cap B) + P(A \cap \bar{B}) \\
       &= P(B)P(A|B) + P(\bar{B})P(A|\bar{B})
\end{align*}
\end{appendices}

\end{document}
